% -*- coding: utf-8 -*-

\chapter{南开大学学位论文格式宏包 NKThesis 使用说明} \label{chpt:A}

\section{系统要求}

模板仅在 TexLive 2020/2021/2022 下测试\XeLaTeX 通过。
对于其它 TeX 发行版可能需要做个别修改。
本模板改编自孙老师(http://222.30.48.141/$\sim$sunwch)的版本,
格式参照南开大学2019年毕业论文格式。

\section{NKThesis 使用说明}

本模板可以使用以下两种方式编译:
\begin{enumerate}
 \item \XeLaTeX + BibTex [推荐]
 \item PDF\LaTeX + BibTex
\end{enumerate}


我们建议您使用\XeLaTeX 编译。与PDF\LaTeX 相比,\XeLaTeX  编译长文档的速度更快,
编译一篇一百多页的论文只需几秒的时间(SL9400 @ 1.86GHz)。

在改变编译方式前应先删除 *.toc 和 *.aux 文件,
因为不同编译方式产生的辅助文件格式可能并不相同。


在Windows系统中可以直接使用PDF\LaTeX。
如果想在Overleaf中使用,请自行上传相关字体文件到根目录:
simfang.ttf,SimSun.ttc,simkai.ttf,SimHei.ttf。


本模板用到 宋体、楷体、仿宋、黑体四种字体. 若需重新配置字体, 请修改 NKTfonts.cfg。
对于 Linux/Mac, 可能需要设置环境变量 OSFONTDIR, 具体内容请参考 texmf.cnf。



\section{引用章节号}
\label{sec:ex:A}

引用章节号请参考如下格式: \ref{chpt:A}\ref{sec:ex:A}.


\section{中英文间隔}

使用 \XeLaTeX\ 编译时,会自动在中英文转换时添加必要的空格。 使用 [PDF]\LaTeX\
编译时仅忽略中文之间的空格,而中英文之间的空格予以保留。
因此,不管何种编译方式,您都不需要在中英文间添加 $\tilde{}$ 以获得额外的空格。例如,

这是 English 中文 $x=y$ 测试

这是English中文$x=y$测试

可以看出,以上两行用 \XeLaTeX\ 编译的结果是相同的。


\section{NKThesis 预调用的宏包}

NKThesis 已经调用以下宏包,您无须重新调用。

\begin{center}
\tablecaption{NKThesis 预调用的宏包}
\begin{tabular}{l|l}
\hline
编译方式 & 调用的宏包\\ \hline
\XeLaTeX & xeCJK, CJKnumb, graphicx, mathptmx \\ \hline
[PDF]\LaTeX & CJK, CJKnumb, uniGBK, graphicx, mathptmx \\
\hline
\end{tabular}
\end{center}


\section{图表}

插图的例子:

\begin{center}
\includegraphics[viewport=0 0 2984 969,width=40mm]{nankaidaxue.pdf}
\figurecaption{南开大学}
\end{center}

\section{字体}

一般情况下, 您不需要显式地设置字体. 如果确实需要, 请使用以下命令

\begin{verbatim}
宋体:  \rmfamily\upshape 或 \songti
黑体:  \bfseries 或 \heiti
楷体:  \itshape  或 \kaiti
仿宋:  \ttfamily 或 \fangsong
加粗:  \jiacu
\end{verbatim}


\section{参考文献} \label{manual:ref}
参考文献引用:
\cite{ChenCheChen2001,Nadkarni-1992,Hua-Wang-1973}
文献\cite{ZhuKeZhen,Huo}\cite{Timoshenko,Zhang-Wang}
\cite[Theorem 2.1]{ZhuKeZhen}

\subsection{录入参考文献}

本模板采用 bibtex 宏包管理参考文献。如果你对此不熟悉,可以
\begin{enumerate}
\item 参考宏包使用说明,或者
\item 手工排版参考文献,然后参考 nkthesis.bib 最后 3 条的格式录入。
\end{enumerate}



\section{一些建议}
\subsection{关于分数的写法}


\LaTeX 提供宏命令\verb+\frac+, 用以打印分数. 为使得版面整齐, 该命令的使用应遵循以下原则:

\begin{enumerate}
\item 仅在分行表达式中使用,
\item 不嵌套使用,
\item 不在上下标中使用.
\end{enumerate}

也就是说, 行内表达式和上下标中出现分数时一律用 $a/b$表示, 如
$(x+2)/((3x^2+4)(7+y))$. 下面是居中表达式:

\[
 x^2 = y^{1/2} +3.
\]

多行表达式: 尽量在加、减、乘、等号前换行. 在乘号前换行时,
下一行首用 \string\times:
\def\iint{\mathop{\int\!\!\!\int}}\def\calG{\mathcal G}
\begin{eqnarray}
&&\left|(W_{\psi_1}f)(a,b)-(W_{\psi_1}f)(a_j,b_{j,k})\right|^{2}\nonumber\\
&=&\frac{1}{C^{2}_{\varphi}}\Bigg|\iint_{\calG} (W_{\varphi}f)(s,t) \nonumber\\
&&\qquad\times \Bigg( (W_{\psi_1}\varphi)\left(\frac{a}{s},
\frac{b-t}{s}\right)
     -(W_{\psi_1}\varphi)\left(\frac{a_{j}}{s}, \frac{b_{j,k}-t}{s}\right)\Bigg)
  \frac{dsdt}{s^{d+1}}\Bigg|^2 \nonumber\\
&\le& \frac{1}{C^2_{\varphi}} \iint_{\calG} |(W_{\varphi}f)(s,t)|^2 \nonumber\\
&&\qquad \times\left| (W_{\psi_1}\varphi)\left(\frac{a}{s},
\frac{b-t}{s}\right)
    -(W_{\psi_1}\varphi)\left(\frac{a_{j}}{s}, \frac{b_{j,k}-t}{s}\right)\right|
   \frac{dsdt}{s^{d+1}}  \nonumber\\
&&\qquad \times   \iint_{\calG}\!
 \left|(W_{\psi_1}\varphi)\left(\frac{a}{s}, \frac{b-t}{s}\right)
    -(W_{\psi_1}\varphi)\left(\frac{a_{j}}{s}, \frac{b_{j,k}-t}{s}\right)\right|
 \frac{ ds dt}{s^{d+1}} \nonumber\\
&=& \frac{1}{C^2_{\varphi}} ....  \label{eq:a0}
\end{eqnarray}


\subsection{标点}
科技文献中一般用半角标点, 请参考《中国科学》发表的论文.

如果使用全角标点, 可以使用
\begin{verbatim}
  \punctstyle{<style>}
\end{verbatim}
选择标点样式, 有效值为
\begin{verbatim}
  quanjiao (所有标点符号占一个汉字宽度,
            相邻两个标点占一个半汉字宽度)
  banjiao  (所有标点符号占半个汉字宽度)
  hangmobanjiao (所有标点符号占一个汉字宽度,行末行首半角)
  kaiming  (句号、叹号、问号占一个汉字宽度,其他标点占半个汉字宽度)
\end{verbatim}
缺省为全角式。注意:不论选择哪种样式,都提供行末对齐(margin kerning)功能。



\begin{Theorem} \label{thm:latex}
\LaTeX 的输出是最完美的.
\end{Theorem}

先证明一个引理
\begin{Lemma} \label{thm:tex}
\TeX 文件在不同操作系统下的排版结果完全一致.
\end{Lemma}

\begin{proof}
这是证明.
\end{proof}


\begin{proof}[定理~\ref{thm:latex}的证明]
显然是错的.
\end{proof}

单个带编号的表达式
\begin{equation}\label{eq:a1}
x=y+z
\end{equation}

单个不带编号的表达式
\[
y=x-z.
\]

不带编号的多行表达式
\begin{eqnarray*}
x&=&y+z \\
 &=&z-s\\
 &<& 3. \\
 && \mbox{一些注释}
\end{eqnarray*}

带编号的多行表达式
\begin{eqnarray}
 x&=& y-z, \label{eq:aa1}\\
 y&=& x+z, \nonumber \\
 z&=&y-x. \label{eq:aa2}
\end{eqnarray}



引用:   定理\ref{thm:latex}的推论是什么呢?
方程式编号:  由(\ref{eq:a1})(\ref{eq:aa2})式.

\subsection{列举环境:  enumerate}

环境 enumerate 已经被改写,增加了一个可选参数[字符串], 用以控制所进。例如,
\begin{verbatim}
  \begin{enumerate}
    \item This is an example.
    \item This is an example.
      \begin{enumerate}
      \item This is an example.
      \item This is an example.
    \end{enumerate}
  \end{enumerate}
  \begin{enumerate}[Mn]% 字符串"Mn"的宽度为增加的缩进。
                       % 缺省值为 [M]
    \item This is an example.
    \item This is an example.
      \begin{enumerate}[Mnn]% 字符串"Mnn"的宽度为增加的缩进。
      \item This is an example.
      \item This is an example.
    \end{enumerate}
  \end{enumerate}
\end{verbatim}
的输出为
  \begin{enumerate}
    \item This is an example.
    \item This is an example.
      \begin{enumerate}
      \item This is an example.
      \item This is an example.
    \end{enumerate}
  \end{enumerate}
  \begin{enumerate}[Mn]% 字符串"Mn"的宽度为增加的所进。
                       % 缺省值为 [M]
    \item This is an example.
    \item This is an example.
      \begin{enumerate}[Mnn]% 字符串"Mnn"的宽度为增加的所进。
      \item This is an example.
      \item This is an example.
    \end{enumerate}
  \end{enumerate}

\clearpage

\section{\TeX\ 简介}

以下内容是 milksea@bbs.ctex.org 撰写的关于\TeX\ 的简单介绍。
注意这不是一个入门教程,不讲 \TeX\ 系统的配置安装,也不讲具体的 \LaTeX\ 代码。
这里仅仅试图以一些只言片语来解释:
进入这个门槛之前新手应该知道的注意事项,以及遇到问题以后该去如何解决问题。

\subsection{什么是Latex,我是否应该选择它}

\TeX\ 是最早由高德纳(Donald Knuth)教授创建的一门标记式宏语言,
用来排版科技文章,尤其擅长处理复杂的数学公式。\TeX\ 同时也是处理这一语言的排版软件。
\LaTeX\ 是 Leslie Lamport 在 \TeX\ 基础上按内容/格式分离和模块化等思想建立的一集 \TeX\ 上的格式。

\TeX\ 本身的领域是专业排版(即方正书版、InDesign 的领域),
但现在 TeX/LaTeX 也被广泛用于生成电子文档甚至幻灯片等,\TeX\ 语言的数学部分
偶尔也在其他一些地方使用。但注意 \TeX\ 并不适用于文书处理(MS Office 的领域,以前和现在都不是)。

选择使用 \TeX/\LaTeX\ 的理由包括:
\begin{itemize}
\item 免费软件;
\item 专业的排版效果;
\item 是事实上的专业数学排版标准;
\item 广泛的西文期刊接收甚或只接收 LaTeX 格式的投稿;
\item[] ……
\end{itemize}
不选择使用 \TeX/\LaTeX\ 的理由包括:
\begin{itemize}
\item 需要相当精力学习;
\item 图文混合排版能力弱;
\item 仅流行于数学、物理、计算机等领域;
\item 中文期刊的支持较差;
\item[] ……
\end{itemize}

请尽量清醒看待网上经常见到的关于 \TeX\ 与其他软件的优劣比较和口水战。在选择使用或离开之前,请先考虑
\TeX\ 的应用领域,想想它是否适合你的需要。

\def\AAAA{}

\subsection{我该用什么编辑器?}

编辑器功能有简有繁,特色不一,因人而易。
基本功能有语法高亮、方便编译预览就很好了,扩充功能和定制有无限的可能。
初学者强烈推荐免费的VSCode + TexLive。
VSCode支持多种插件,如LaTex Workshop插件支持LaTeX的编译,
Git Project Manager支持版本同步等。


\subsection{我该去哪里寻找答案?}

0、绝对的新手,先读完一本入门读物,了解基本的知识。

1、无论如何,先读文档!绝大部分问题都是文档可以解决的。

2、再利用 Google 搜索,利用(bbs.ctex.org)版面搜索。

3、清楚、聪明地提出你的问题。


\subsection{我应该看什么 LaTeX 读物?}

这不是一个容易回答的问题,因为有许多选择,也同样有许多不合适的选择。
这里只是选出一个比较好的答案。更多更详细的介绍可以在版面和网上寻找(注意时效)。

近两年 \TeX\ 的中文处理发展很快,目前没有哪本书在中文处理方面给出一个最新进展的合适综述,
因而下面的介绍也不主要考虑中文处理。

\begin{enumerate}
\item 我可以阅读英文

\begin{enumerate}
\item 我要迅速入门:ltxprimer.pdf (LaTeX Tutorials: A Primer, India TUG)
\item 我要系统学习:A Guide to LaTeX, 4th Edition, Addison-Wesley
      有机械工业出版社的影印版(《LaTeX实用教程》)
\item 我要深入学习:要读许多书和文档,TeXbook 是必读的
\item 还有呢?去读你使用的每一个宏包的说明文档
\item 还有许多专题文档,如讲数学公式、图形、表格、字体等
\end{enumerate}

\item 我更愿意阅读中文
\begin{enumerate}
\item 我要迅速入门:lnotes.pdf (LaTeX Notes, 1.20, Alpha Huang)
\item 我要系统学习:《LaTeX2ε 科技排版指南》,邓建松(电子版)
 如果不好找,看《LaTeX 入门与提高》第二版,陈志杰等
\item 我要深入学习:TeXbook0.pdf (特可爱原本,TeXbook 的中译,xianxian)
\item 还有呢?英语,绝大多数 TeX 资料还是英文的
\end{enumerate}
\end{enumerate}


\subsection{插图格式}

常见的图片格式都支持,如PDF、PNG、JPG格式的图形。



\subsection{作图}

目前已经有很多优秀的\LaTeX\ 作图宏包,如 pgf/Tikz 和 pstricks,
两者都具有强大的作图能力。

\chapter{说明}



本模板参照南开大学学位论文写作规范编写,
仅仅提供了论文的基本格式,包括章节标题和正文字体、字号等等的设置。



您自愿使用这个模板。
提供本模板的目的是为了给您的论文写作带来方便,然而,
作者不保证这个模板完全符合学校的要求,也不对由此产生的任何后果负责。
如果您不同意这些条款,请不要使用这个模板。


参考文献的录入请参考\ref{manual:ref}。 